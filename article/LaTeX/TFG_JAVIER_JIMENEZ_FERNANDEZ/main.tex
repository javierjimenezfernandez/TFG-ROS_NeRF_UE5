% Editor de fórmulas matemáticas
%https://www.codecogs.com/latex/eqneditor.php?lang=es-es

%%%%%%%%%%%%%%%%% - PREÁMBULO - %%%%%%%%%%%%%%%%%


% ---------- Metadatos del documento ----------- %
\title{Digital Twins con control telemático inmersivo mediante el uso de realidad virtual generada por NeRF y 3DGS en tareas de rescate}
\author{Javier Jiménez Fernández}
\date{Septiembre 2024}
% ---------------------------------------------- %


% --------- Composición de la página ---------- %
% Para el texto escrito se utilizará siempre hojas blancas de tamaño A4 (297 x 210 mm) que estarán escritas, preferentemente, por las dos caras. Existen numerosas fuentes válidas aunque se recomienda utilizar “Arial” (tamaño 11 puntos) o “Times New Roman” (tamaño 12 puntos).
\documentclass[a4paper, 12pt, spanish, twoside]{article}
% Los márgenes tanto derecho, como izquierdo, superior e inferior serán de 25 mm.
\usepackage[top=2.5cm,bottom=2.5cm,left=2.5cm,right=2.5cm]{geometry}
\raggedbottom
% ---------------------------------------------- %



% ------------- Paquetes generales ------------- %
\usepackage[utf8]{inputenc}
\usepackage[spanish,es-tabla]{babel}
\usepackage{float}
\usepackage{caption}
% ---------------------------------------------- %



% ------------ Paquetes específicos ------------ %
\usepackage{pdfpages} % Para insertar la portada en formato PDF.
\usepackage{pdflscape}  % Para colocar páginas en formato apaisado.
\usepackage{graphicx} % Para insertar imágenes.
\graphicspath{{imagenes/}} % Configuración del paquete graphicx, imágenes en la carpeta images
\usepackage{wrapfig} % Para posicionar imágenes alrededor del texto.
\usepackage[hidelinks]{hyperref} % Para urls.
% ---------------------------------------------- %



% ---------------- Numeración ------------------ %
\counterwithin{table}{section} % Se numeran las tablas con respecto al capítulo en el que se encuentran.
\counterwithin{figure}{section} % Se numeran las figuras con respecto al capítulo en el que se encuentran.
\counterwithin{equation}{section} % Se numeran las ecuaciones con respecto al capítulo en el que se encuentran.
% ---------------------------------------------- %



% ------------- Página en blanco ----------------%
% Se define un comando (\blankpage) para insertar una página totalmente en blanco (sin número de página, encabezado y pie de página):
\usepackage{afterpage}
\newcommand\blankpage{%
    \null
    \thispagestyle{empty}%
    \newpage}
% ---------------------------------------------- %



% ----------- Formato de los párrafos -----------%
% El interlineado recomendado es el sencillo, con renglón libre entre párrafos. El texto debe ser justificado en ambos márgenes, de manera que quede constante la longitud de cada línea del párrafo.
% Se define el formato de los párrafos:
\setlength{\parindent}{0pt} % Se elimina la sangría en comienzo de párrafo (0pt).
\setlength{\parskip}{1em} % Se define el espacio entre dos párrafos (1em).
% ---------------------------------------------- % 



% ---------- Leyendas y pies de foto ----------- %
\captionsetup{justification=centering} % Justificación de las leyendas y pies de foto
% ---------------------------------------------- %



% -------------- Título adicional -------------- %
% Se añade una profundidad adicional a los títulos (profundidad 4):
\usepackage{titlesec}
\setcounter{secnumdepth}{4} % Se fija en 4 la profundidad de numeración de títulos.
\setcounter{tocdepth}{4} % Se fija en 4 la profundidad de títulos incluidos en el índice.
% Se modifica el formato de \paragraph (título de profundidad 4) para adaptarlo al formato del resto de títulos:
\titleformat{\paragraph}
{\normalfont\normalsize\bfseries}{\theparagraph}{1em}{}
\titlespacing*{\paragraph}
{0pt}{3.25ex plus 1ex minus .2ex}{1.5ex plus .2ex} 
% ---------------------------------------------- %  



% --------- Encabezado y pie de página -------- %
% Es conveniente la utilización de encabezados y pies de páginas que proporcionen información auxiliar para la mejor lectura y compresión del proyecto, separados del cuerpo del texto por una línea continua y a 15 mm de los márgenes superior e inferior. A título orientativo, en las páginas pares se incluirá en el encabezamiento y justificado a la izquierda el apartado correspondiente (p.ej. Resultados) y en el pie se indicará justificado a la izquierda el número de página y justificado a la derecha “Escuela Técnica Superior de Ingenieros Industriales (UPM)”. En las páginas impares en el encabezamiento figurará justificado a la derecha el título entero del TFG o TFM o un título abreviado para que no exceda de una línea y en el pie se indicará justificado a la izquierda el nombre y apellidos del alumno y justificado a la derecha el número de página.
% El encabezado y pie de página forman parte del paquete fancyhdr:
\usepackage{fancyhdr}
\fancyhf{}
\pagestyle{fancy}

% Se ajusta el tamaño de fuente para el encabezado y pie de página (9pt)
\fancyhf{\fontsize{2}{12}\selectfont}

% Contenido del encabezado (\fancyhead):
\fancyhead[RO]{Digital Twins, control telemático y RV generada por NeRF y 3DGS en tareas de rescate} % Texto que se coloca en el encabezado de las páginas impares (O -> 'Odd', o impar) a la izquierda (R -> 'Odd')
\fancyhead[LE]{\nouppercase{\rightmark}} % Texto que se coloca en el encabezado de las páginas pares (E -> 'Even', o par) a la izquierda (L -> 'Left'). \rightmark se utiliza para insertar automáticamente el título de la sección correspondiente, y \nouppercase para que no aparezca todo en mayúsculas (formato por defecto de \rightmark).

% Contenido del pie de página (\fancyfoot):
\fancyfoot[RE]{Escuela  Técnica  Superior  de  Ingenieros  Industriales  (UPM)} % Texto que se coloca en el pie de página de las páginas pares (E -> 'Even', o par) a la derecha (R -> 'Right')
\fancyfoot[LO]{Javier Jiménez Fernández} % Texto que se coloca en el pie de página de las páginas impares (O -> 'Odd', o impar) a la izquierda (L -> 'Left')
\fancyfoot[LE,RO]{\thepage} % El número de página (\thepage) se coloca a la izquierda en las páginas pares y a la derecha en las impares.

% Se indica que sólo se quiere incorporar en \rightmark (utilizado más arriba) el título de la sección (y no de las subsecciones, subsubsecciones, etc.):
\renewcommand{\sectionmark}[1]{\markright{\thesection. #1}}
\renewcommand{\subsectionmark}[1]{}

% Formato de la línea de separación horizontal:
\renewcommand{\headrulewidth}{0.5pt} % Ancho de la línea del encabezado.
\renewcommand{\footrulewidth}{0.5pt} % Ancho de la línea del pie de página.
% ---------------------------------------------- % 



% ----------- Fragmentos de código ------------- %
% El paquete utilizado para insertar fragmentos de código en el documento es listings. En el presente bloque del preámbulo se definen ciertos parámetros de listings con el objetivo de adaptar dicho paquete a código escrito en Python.

\usepackage{listings} % Paquete para insertar código. 
\usepackage{xcolor} % Paquete para definir colores.

% Se definen los distintos colores que se utilizan para resaltar ciertos elementos del código:
\definecolor{codegreen}{rgb}{0.04314,0.6745,0.07843} % Verde.
\definecolor{codegray}{rgb}{0.5,0.5,0.5} % Gris.
\definecolor{codered}{rgb}{0.5373,0.02745,0.06275} % Rojo.
\definecolor{codeblue}{rgb}{0.071,0.0258,0.9882} % Azul.
\definecolor{codepurple}{rgb}{0.6,0.02745,0.5961} % Morado.

% Se define el color de fondo:
\definecolor{backcolour}{rgb}{0.95,0.95,0.92} % Gris oscuro.

% Se define el valor de ciertos parámetros de listings para adaptar dicho paquete a código escrito en Python:
\lstdefinestyle{mystyle}{
    % - General:
    language=C++, % Lenguaje de programación.
    basicstyle=\ttfamily\footnotesize, % Tipografía y tamaño de fuente.
    % - Colores de los distintos elementos del código:
    backgroundcolor=\color{backcolour}, % Color de fondo.  
    commentstyle=\color{codegray}, % Color de los comentarios.
    keywordstyle=\color{codeblue}, % Color de las palabras clave por defecto.
    stringstyle=\color{codegreen}, % Color de los "string"
    % - Palabras clave:
    deletekeywords={print}, % Se elimina "print" del conjunto de palabras clave para posteriormente asignarle el color morado.
    keywordstyle={[2]\ttfamily\color{codeblue}},
    keywords=[2]{as}, % Se añaden las palabras clave de color azul.
    keywordstyle={[3]\ttfamily\color{codepurple}},
    keywords=[3]{True, False, ttk, list, None, dict, zip, range, len, print, float, sum}, % Se añaden las palabras clave de color morado.
    keywordstyle={[4]\bfseries\ttfamily},
    keywords=[4]{_read_excel}, % Se añaden las palabras clave en negrita.
    emph={MyClass,__init__}, % Se añaden las palabras clave enfatizadas.   
    % - Números de línea:
    numberstyle=\tiny\color{codegray}, % Tamaño de fuente y color de los números de línea.
    numbers=left, % Se colocan los números de línea en el lado izquierdo.                 
    numbersep=5pt, % Separación horizontal de los números de línea.
    % - Saltos a la línea, espacios, indentación:
    breaklines=true, % Permitir saltos a la línea. 
    breakatwhitespace=true, % Saltar a la línea únicamente al encontrar espacios.
    postbreak = \mbox{{$\hookrightarrow$}\space}, % Se añade una flecha al cambiar de línea.
    showspaces=false, % No mostrar los espacios. 
    showstringspaces=false, % No mostrar los espacios en los "string".
    keepspaces=true, % Mantener los espacios presentes en el código. 
    tabsize=2, % Tamaño de indentación.
    % - Título:
    captionpos=b % Posición del título del fragmento de código (b=bottom - abajo).
} 
\lstset{style=mystyle} % Se asocia el estilo de listings al estilo que acaba de definirse ("mystyle")

% Se realizan una serie de operaciones complementarias con el paquete listings (su comprensión no es necesaria para manejar dicho paquete):
\makeatletter
\def\lst@OpLiteratekey#1\@nil@{\let\lst@ifxopliterate\lst@if
                             \def\lst@opliterate{#1}}
\lst@Key{opliterate}{}{\@ifstar{\lst@true \lst@OpLiteratekey}
                             {\lst@false\lst@OpLiteratekey}#1\@nil@}
\lst@AddToHook{SelectCharTable}
    {\ifx\lst@opliterate\@empty\else
         \expandafter\lst@OpLiterate\lst@opliterate{}\relax\z@
     \fi}
\def\lst@OpLiterate#1#2#3{%
    \ifx\relax#2\@empty\else
        \lst@CArgX #1\relax\lst@CDef
            {}
            {\let\lst@next\@empty
             \lst@ifxopliterate
                \lst@ifmode \let\lst@next\lst@CArgEmpty \fi
             \fi
             \ifx\lst@next\@empty
                 \ifx\lst@OutputBox\@gobble\else
                   \lst@XPrintToken \let\lst@scanmode\lst@scan@m
                   \lst@token{#2}\lst@length#3\relax
                   \lst@XPrintToken
                 \fi
                 \let\lst@next\lst@CArgEmptyGobble
             \fi
             \lst@next}%
            \@empty
        \expandafter\lst@OpLiterate
    \fi}

\lstset{ 
    literate={á}{{\'a}}1 {é}{{\'e}}1 {í}{{\'i}}1 {ó}{{\'o}}1 {ú}{{\'u}}1
  {Á}{{\'A}}1 {É}{{\'E}}1 {Í}{{\'I}}1 {Ó}{{\'O}}1 {Ú}{{\'U}}1
  {à}{{\`a}}1 {è}{{\`e}}1 {ì}{{\`i}}1 {ò}{{\`o}}1 {ù}{{\`u}}1
  {À}{{\`A}}1 {È}{{\'E}}1 {Ì}{{\`I}}1 {Ò}{{\`O}}1 {Ù}{{\`U}}1
  {ä}{{\"a}}1 {ë}{{\"e}}1 {ï}{{\"i}}1 {ö}{{\"o}}1 {ü}{{\"u}}1
  {Ä}{{\"A}}1 {Ë}{{\"E}}1 {Ï}{{\"I}}1 {Ö}{{\"O}}1 {Ü}{{\"U}}1
  {â}{{\^a}}1 {ê}{{\^e}}1 {î}{{\^i}}1 {ô}{{\^o}}1 {û}{{\^u}}1
  {Â}{{\^A}}1 {Ê}{{\^E}}1 {Î}{{\^I}}1 {Ô}{{\^O}}1 {Û}{{\^U}}1
  {Ã}{{\~A}}1 {ã}{{\~a}}1 {Õ}{{\~O}}1 {õ}{{\~o}}1
  {œ}{{\oe}}1 {Œ}{{\OE}}1 {æ}{{\ae}}1 {Æ}{{\AE}}1 {ß}{{\ss}}1
  {ű}{{\H{u}}}1 {Ű}{{\H{U}}}1 {ő}{{\H{o}}}1 {Ő}{{\H{O}}}1
  {ç}{{\c c}}1 {Ç}{{\c C}}1 {ø}{{\o}}1 {å}{{\r a}}1 {Å}{{\r A}}1
  {€}{{\euro}}1 {£}{{\pounds}}1 {«}{{\guillemotleft}}1
  {»}{{\guillemotright}}1 {ñ}{{\~n}}1 {Ñ}{{\~N}}1 {¿}{{?`}}1
  {º}{{\textordmasculine}}1}

\lstset{opliterate=
   *{0}{{{\color{codered}0}}}1 {1}{{{\color{codered}1}}}1 
   {2}{{{\color{codered}2}}}1 {3}{{{\color{codered}3}}}1 
   {4}{{{\color{codered}4}}}1 {5}{{{\color{codered}5}}}1 
   {6}{{{\color{codered}6}}}1 {7}{{{\color{codered}7}}}1 
   {8}{{{\color{codered}8}}}1 {9}{{{\color{codered}9}}}1}

\DeclareCaptionType{code}[Código][ÍNDICE DE CÓDIGOS] % Se define el entorno "Código" (de forma que al introducir un fragmento de código en el documento aparezca como: Código 1.1: ...), y la lista con los distintos códigos ("Índice de códigos").
\counterwithin{code}{section} % Se numeran los códigos con respecto al capítulo en el que se encuentran.
% ---------------------------------------------- % 



% --------------- Bibliografía ----------------- %
% El manejo de la bibliografía se realiza mediante el paquete biblatex:
\usepackage[backend=biber, style=authoryear, sorting=nyt, citestyle=authoryear, maxcitenames=2, maxbibnames=5, giveninits=true, uniquename=init]{biblatex} 

% Los distintos parámetros que aparecen en la línea anterior corresponden a las siguientes características de la bibliografía:
% - style: la manera en la que aparecen las referencias en la bibliografía. En este caso se opta por "authoryear", pero existen múltiples estilos posibles que se resumen en la siguiente guía: https://www.overleaf.com/learn/latex/biblatex_bibliography_styles.
% - sorting: orden en el que aparecen las distintas referencias en la bibliografía. En este caso se opta por ordenarlas en primer lugar por el apellido del primer autor, en segundo lugar por el año de publicación, y por último por el título de la publicación (nyt=name-year-title)
% - citestyle: elementos y orden de dichos elementos de una referencia al citarla en el documento. En este caso se escoge "authoryear" para que aparezca en primer lugar el apellido del autor (o de los autores) y en segundo lugar el año de publicación. Existe gran variedad de opciones en cuanto al parámetro citestyle que se resumen en: https://www.overleaf.com/learn/latex/biblatex_citation_styles.
% maxcitenames: máximo número de autores que aparecen al citar una referencia en el documento. Al escoger un valor de 2 para este parámetro se pueden dar los siguientes casos: un único autor -> (autor, año), dos autores -> (autor 1 y/e autor 2, año), tres o más autores -> (autor 1 et al., año).
% maxbibnames: parámetro idéntico al anterior pero para la bibliografía en lugar de las citas.
% giveinits y uniquename: para mostrar únicamente las iniciales de los nombres de los autores.

% Se importa el paquete csquotes para citar las referencias a lo largo del documento:
\usepackage{csquotes} 

% Se realizan una serie de operaciones para adaptar la bibliografía al estilo deseado (coma entre autor y año al citar una referencia, idioma castellano, etc.):
\DeclareNameAlias{sortname}{family-given}
\renewcommand*{\nameyeardelim}{\addcomma\space}
\setlength\bibitemsep{\baselineskip}
\DefineBibliographyStrings{spanish}{%
  andothers = {et\addabbrvspace al\adddot}
}

\makeatletter

\newrobustcmd*{\parentexttrack}[1]{%
  \begingroup
  \blx@blxinit
  \blx@setsfcodes
  \blx@bibopenparen#1\blx@bibcloseparen
  \endgroup}

\AtEveryCite{%
  \let\parentext=\parentexttrack%
  \let\bibopenparen=\bibopenbracket%
  \let\bibcloseparen=\bibclosebracket}

\makeatother

\addbibresource{bibliografia.bib}
% ---------------------------------------------- % 


% ----------- Glosario y acrónimos ------------- %
% Para el glosario y los acrónimos, que aparezcan en el índice y que sean una subsubsección
\usepackage[section=subsubsection,acronym,toc]{glossaries}

\makeglossaries
\loadglsentries{glosario}
\loadglsentries{acronimos}

% ---------------------------------------------- %



%%%%%%%%%%%% - INICIO DEL DOCUMENTO - %%%%%%%%%%%%

\begin{document}

%%%%%%%%%%%%%%%%%%%%%%%%%%%%%%%%%%%%%%%%%%%%%%%%%%



%%%%%%%%%%%%%%%%%%% - PORTADA - %%%%%%%%%%%%%%%%%%

% Se comienza una página nueva sin formato (sin número de página y sin encabezado/pie de página), ya que sólo incorpora la portada:
\newpage
\thispagestyle{empty}

% La portada se inserta mediante el comando \includepdf seguido del archivo PDF correspondiente (que se ajusta automáticamente a las dimensiones de la página):
\includepdf[landscape]{PORTADA_TFG_JAVIER_JIMENEZ_FERNANDEZ.pdf}

% Termina la página actual y hace que se impriman todas las figuras y tablas que han aparecido hasta ahora en la entrada:
\clearpage
% La página detrás de la portada va vacía
\blankpage

%%%%%%%%%%%%%%%%%%%%%%%%%%%%%%%%%%%%%%%%%%%%%%%%%%



%%%%%%%%%%%%%%%%%%%%%%%%%%%%%%%%%%%%%%%%%%%%%%%%%%
% --------- Numeración primeras páginas -------- %
% Las páginas anteriores al contenido del TFG/TFM (previas a la introducción) suelen numerarse de forma distinta a las del cuerpo del informe, en este caso en números romanos:
\pagenumbering{roman}
% ---------------------------------------------- % 
%%%%%%%%%%%%%%%%%%%%%%%%%%%%%%%%%%%%%%%%%%%%%%%%%%



%%%%%%%%%%%%%%%%% - PORTADA CAR - %%%%%%%%%%%%%%%%

\newpage
\thispagestyle{empty}

%\includepdf{}
Aquí va la portada del CAR 

logotipos 

UPM

ETSII

GITI 

Logotipo del CAR 

TFG

Digital Twins con control telemático inmersivo mediante el uso de realidad virtual generada por NeRF y 3DGS en tareas de rescate

Autor: Javier Jiménez Fernández

Tutor Académico: Antonio Barrientos Cruz

Cotutor:  Christyan Mario Cruz Ulloa

Septiembre, 2024

% Termina la página actual y hace que se impriman todas las figuras y tablas que han aparecido hasta ahora en la entrada:
\clearpage
% La página detrás de la portada del CAR va vacía
\afterpage{\blankpage}

%%%%%%%%%%%%%%%%%%%%%%%%%%%%%%%%%%%%%%%%%%%%%%%%%%



%%%%%%%%%%%%%%%%%%% - CITA - %%%%%%%%%%%%%%%%%%%%%

% Se comienza una página nueva sin formato (sin número de página y sin encabezado/pie de página), ya que sólo incorpora la cita:
\newpage
\thispagestyle{empty}

\begin{flushright} % Se alinea el texto en el lado derecho de la página.
\vspace*{5cm} % Se añade un espacio vertical de 5cm para situar la cita en ~1/3 de la página.

\textit{“El éxito en la vida no se mide por lo que has logrado, sino por los obstáculos que has superado.”} 

\medskip % Salto a la línea de tamaño medio (existen \smallskip, \medskip y \bigskip)
- Robert Baden-Powell 

\end{flushright}

% Termina la página actual y hace que se impriman todas las figuras y tablas que han aparecido hasta ahora en la entrada:
\clearpage
\afterpage{\blankpage} % Se añade una página en blanco después de la cita.

%%%%%%%%%%%%%%%%%%%%%%%%%%%%%%%%%%%%%%%%%%%%%%%%%%



%%%%%%%%%%%%% - AGRADECIMIENTOS - %%%%%%%%%%%%%%%%

% Se comienza una página nueva con formato plano (sin encabezado/pie de página pero con número de página):
\newpage
\thispagestyle{plain}

\section*{AGRADECIMIENTOS} % Se añade un asterisco a \section para que el título no esté numerado.
\addcontentsline{toc}{section}{AGRADECIMIENTOS} % Al utilizar \section* se ha de añadir manualmente el apartado al índice (Table Of Contents, TOC).

Agradezco a \dots

mi madre,
Mandy,
mi amigos,
compañeros de la universidad,
compañeros de trabajo,

Barrientos,
Chrystian,
David,
Jorge,

Fernando Rivas y Volinga

% Termina la página actual y hace que se impriman todas las figuras y tablas que han aparecido hasta ahora en la entrada:
\clearpage
\afterpage{\blankpage} % Se añade una página en blanco después de los agradecimientos.

%%%%%%%%%%%%%%%%%%%%%%%%%%%%%%%%%%%%%%%%%%%%%%%%%%



%%%%%%%%%% - CÓMO LEER ESTE DOCUMENTO - %%%%%%%%%%

\section*{CÓMO LEER ESTE DOCUMENTO} % Se añade un asterisco a \section para que el título no esté numerado.
\markright{CÓMO LEER ESTE DOCUMENTO} % Al utilizar \section* se ha de añadir manualmente el título del apartado al encabezado.
\addcontentsline{toc}{section}{CÓMO LEER ESTE DOCUMENTO} % Al utilizar \section* se ha de añadir manualmente el apartado al índice (Table Of Contents, TOC).

En el presente artículo se hace referencia a multitud de términos y acrónimos técnicos específicos de la tecnología de la que es objeto. Se anima al lector a hacer uso de las secciones Glosario y Acrónimos durante su lectura.

Por comodidad, se ha realizado una referencia cruzada de todas las menciones a términos y acrónimos definidos en los glosarios. Gracias a ello, en la versión en PDF, clicando sobre sus apariciones a lo largo del documento se redireccionará directamente a su definición. 

Este documento consta de un Resumen Ejecutivo o abstract a continuación de esta sección que resume brevemente la investigación de este trabajo. 

Después se encuentra el Índice o tabla general de contenidos y, justo a continuación, se incluyen los Índices de Tablas, Figuras y Códigos de este documento. 

Para seguir el lector encontrará una Introducción sobre cómo dio comienzo este trabajo, un resumen del Estado del Arte en el campo de la presente investigación, los Objetivos planteados, la Metodología utilizada durante el desarrollo de este trabajo y las Herramientas Utilizadas. 

El grueso del trabajo lo constituyen las secciones de Implementación, Resultados y Conclusiones. En Implementación se incluye todo el camino recorrido por esta investigación hasta llegar al resultado final.

Por último, el lector podrá encontrar la Bibliografía y los Anexos. En los anexos se encuentran la Evaluación de Impactos, la cual incluye la contribución a los Objetivos de Desarrollo Sostenible (ODS), la Planificación temporal y presupuestos, el Glosario y Acrónimos y para terminar las Licencias de uso utilizadas durante esta investigación.

% Termina la página actual y hace que se impriman todas las figuras y tablas que han aparecido hasta ahora en la entrada:
\clearpage
\afterpage{\blankpage} % Se añade una página en blanco después de esta sección.

%%%%%%%%%%%%%%%%%%%%%%%%%%%%%%%%%%%%%%%%%%%%%%%%%%



%%%%%%%%%%%%%% - RESUMEN EJECUTIVO - %%%%%%%%%%%%%

\newpage
\section*{RESUMEN EJECUTIVO} % Se añade un asterisco a \section para que el título no esté numerado.
\markright{RESUMEN EJECUTIVO} % Al utilizar \section* se ha de añadir manualmente el título del apartado al encabezado.
\addcontentsline{toc}{section}{RESUMEN EJECUTIVO} % Al utilizar \section* se ha de añadir manualmente el apartado al índice (Table Of Contents, TOC).



\subsection*{Palabras clave} % Se añade un asterisco a \section para que el título no esté numerado.
\addcontentsline{toc}{subsection}{Palabras clave} % Al utilizar \section* se ha de añadir manualmente el apartado al índice (Table Of Contents, TOC).

\subsection*{Códigos UNESCO}
\addcontentsline{toc}{subsection}{Códigos UNESCO} % Al utilizar \section* se ha de añadir manualmente el apartado al índice (Table Of Contents, TOC).

% Termina la página actual y hace que se impriman todas las figuras y tablas que han aparecido hasta ahora en la entrada:
\clearpage
\afterpage{\blankpage} % Se añade una página en blanco después del resumen.

%%%%%%%%%%%%%%%%%%%%%%%%%%%%%%%%%%%%%%%%%%%%%%%%%%



%%%%%%%%%%%%%%%%%%% - ÍNDICE - %%%%%%%%%%%%%%%%%%%

\newpage

\renewcommand*\contentsname{ÍNDICE} % Se modifica el nombre por defecto de la "Table Of Contents" (tabla de contenidos, índice) para pasar a llamarla "ÍNDICE".

\tableofcontents % Se genera el índice de contenidos del documento que incorpora todos los títulos de \section, \subsection y \subsubsection (y también \paragraph, ver capítulo 1), así como los títulos añadidos con \addcontentsline (como el resumen ejecutivo, por ejemplo).

% Termina la página actual y hace que se impriman todas las figuras y tablas que han aparecido hasta ahora en la entrada:
\clearpage
\afterpage{\blankpage} % Se añade una página en blanco después del índice.

%%%%%%%%%%%%%%%%%%%%%%%%%%%%%%%%%%%%%%%%%%%%%%%%%%



%%%%%%%%%%%%%% - ÍNDICE DE TABLAS - %%%%%%%%%%%%%%

\newpage

\renewcommand{\listtablename}{ÍNDICE DE TABLAS} % Se define el nombre del índice de tablas.
\listoftables % Se genera automáticamente el índice con las distintas tablas del documento (entorno \table o \longtable).
\addcontentsline{toc}{section}{ÍNDICE DE TABLAS} % Se añade manualmente el apartado al índice (Table Of Contents, TOC).

% Termina la página actual y hace que se impriman todas las figuras y tablas que han aparecido hasta ahora en la entrada:
\clearpage

%%%%%%%%%%%%%%%%%%%%%%%%%%%%%%%%%%%%%%%%%%%%%%%%%%



%%%%%%%%%%%%% - ÍNDICE DE FIGURAS - %%%%%%%%%%%%%%

\newpage

\renewcommand{\listfigurename}{ÍNDICE DE FIGURAS} % Se define el nombre del índice de figuras.
\listoffigures % Se genera automáticamente el índice con las distintas figuras del documento (entorno \figure).
\addcontentsline{toc}{section}{ÍNDICE DE FIGURAS} % Se añade manualmente el apartado al índice (Table Of Contents, TOC).

% Termina la página actual y hace que se impriman todas las figuras y tablas que han aparecido hasta ahora en la entrada:
\clearpage

%%%%%%%%%%%%%%%%%%%%%%%%%%%%%%%%%%%%%%%%%%%%%%%%%%



%%%%%%%%%%%%%% - ÍNDICE DE CÓDIGOS - %%%%%%%%%%%%%

\newpage

\listofcodes % Se genera automáticamente el índice con los distintos códigos del documento (entorno \code).
\addcontentsline{toc}{section}{ÍNDICE DE CÓDIGOS} % Se añade manualmente el apartado al índice (Table Of Contents, TOC).

% Termina la página actual y hace que se impriman todas las figuras y tablas que han aparecido hasta ahora en la entrada:
\clearpage
\afterpage{\blankpage} % Se añade una página en blanco después del índice de códigos.

%%%%%%%%%%%%%%%%%%%%%%%%%%%%%%%%%%%%%%%%%%%%%%%%%%



%%%%%%%%%%%%%%%%%%%%%%%%%%%%%%%%%%%%%%%%%%%%%%%%%%
% -------------- Numeración normal ------------- %
% Se inicia una nueva página, y se restablece la numeración de las páginas, utilizando esta vez el sistema de numeración estándar (1, 2, 3, 4, ...)
\newpage
\pagenumbering{arabic}
% ---------------------------------------------- % 
%%%%%%%%%%%%%%%%%%%%%%%%%%%%%%%%%%%%%%%%%%%%%%%%%%



%%%%%%%%%%%%%%% - INTRODUCCIÓN - %%%%%%%%%%%%%%%%

\newpage
\section{INTRODUCCIÓN} \label{sec:introduccion}

Siempre existe un desencadenante, una chispa, un comienzo. En este caso fue un vídeo de YouTube. En enero de 2022 el divulgador científico sobre \gls{inteligenciaartificial} \href{https://www.youtube.com/@DotCSV}{@DotCSV} habló del recién publicado paper de \gls{nvidia} Developers \gls{nvlabs} \textit{\gls{instant-ngp}} (\cite{mueller2022instant}). El resultado me impresionó tanto que cuando el curso siguiente me acerqué al \acrfull{robcib} buscando un tutor para mi proyecto final de carrera y Christyan me habló sobre generación de entornos y realidad virtuales al instante supe que era lo que estaba buscando. Le propuse ser quien probase esta nueva tecnología. 

Así comienza una larga historia pues ni fue mi último año de carrera ni esta nueva tecnología era fácil ni simple de implementar. 

A final del curso 2022/2023 había conseguido probar la tecnología de NeRF, pero no había conseguido implementarla de manera satisfactoria con el control de los robots. No lo sabía, pero aún estaba a mitad de camino. Me quedaba madurar más mi investigación y formarme en el uso de nuevas herramientas para llegar al fin donde nos encontramos. 

Hoy me siento orgulloso de lo que he logrado, la paciencia ha dado sus frutos y por fin he hecho el proyecto que quería.  

Tras de mi queda el presente proyecto y toda la documentación para que los que vengan después de mi puedan continuar por este camino.  

% Termina la página actual y hace que se impriman todas las figuras y tablas que han aparecido hasta ahora en la entrada:
\clearpage

%%%%%%%%%%%%%%%%%%%%%%%%%%%%%%%%%%%%%%%%%%%%%%%%%%



%%%%%%%%%%%%%% - ESTADO DEL ARTE - %%%%%%%%%%%%%%%

\newpage
\section{ESTADO DEL ARTE} \label{sec:estado_del_arte}

La tecnología de \gls{renderizado} de \glspl{radiancefield} en tiempo real ha dado un salto importante en los últimos 2 años generando multitud de áreas de investigación e implementaciones que sorprenden por el realismo de los objetos y entornos generados.

Según las publicaciones en papers como \acrfull{nerf} o \acrfull{3dgs} a partir de vídeo o imágenes tomadas por un dispositivo cualquiera, una \gls{redneuronal} es capaz de generar un modelo \acrshort{3d} realista del objeto o entorno filmado o fotografiado en un tiempo reducido de segundos. La inmediatez mencionada no siempre es operativamente cierta como se verá más adelante.

%hablar de la historia de neural radiante fields (NeRF) y como era un concepto teórico que no había sido posible implementar en todo su potencial.

%Contar un poco también de las movidas que conté en mi presentación del año pasado

En enero de 2022 se publica un artículo
%hablar de NVIDIA instant-ngp

%hablar de como se ve su potencial en audiovisuales pero en realidad surge de un entorno cientifico

%casos en los que se usa, hablar del congreso de NVIDIA y de las últimas tendencias

%acercamiento a esta tecnología para digital twins

%hablar de gaussian splatting



% Termina la página actual y hace que se impriman todas las figuras y tablas que han aparecido hasta ahora en la entrada:
\clearpage

%%%%%%%%%%%%%%%%%%%%%%%%%%%%%%%%%%%%%%%%%%%%%%%%%%



%%%%%%%%%%%%%%%%% - OBJETIVOS - %%%%%%%%%%%%%%%%%%

\newpage
\section{OBJETIVOS} \label{sec:objetivos}

Los modelos \acrshort{3d} de la realidad en los que se simulan los \glspl{digitaltwin} deben ser copias lo más fiables posibles del entorno en el que trabajan los robots. 

El objeto de este trabajo es, por un lado, la creación de modelos \acrshort{3d} más realistas que los generados por \gls{fotogrametria} actualmente, especialmente mejorando la resolución y la interpretación de superficies transparentes o reflectantes. 

Para ello se hará uso de nuevas tecnologías en generación de entornos \acrshort{3d}, como \acrfull{nerf} o \acrfull{3dgs}.

Por otro lado, también es objeto del presente trabajo ensayar su robustez y el tiempo de despliegue de la tecnología en situaciones de aplicación realistas, así como la compatibilidad con las líneas de investigación actuales del laboratorio \acrshort{robcib} del \acrshort{car} \acrshort{upm}-\acrshort{csic}.

% Termina la página actual y hace que se impriman todas las figuras y tablas que han aparecido hasta ahora en la entrada:
\clearpage

%%%%%%%%%%%%%%%%%%%%%%%%%%%%%%%%%%%%%%%%%%%%%%%%%%



%%%%%%%%%%%%%%%% - METODOLOGÍA - %%%%%%%%%%%%%%%%%

\newpage
\section{METODOLOGÍA} \label{sec:metodologia}

% Contar bonito que llevo una bitacora y un excel de conteo de horas de trabajo y voy desarrollando manuales de configuración y puesta en marcha de todo lo que voy haciendo por el camino

% Termina la página actual y hace que se impriman todas las figuras y tablas que han aparecido hasta ahora en la entrada:
\clearpage

%%%%%%%%%%%%%%%%%%%%%%%%%%%%%%%%%%%%%%%%%%%%%%%%%%



%%%%%%%%%%% - HERRAMIENTAS UTILIZADAS - %%%%%%%%%%

\newpage
\section{HERRAMIENTAS UTILIZADAS} \label{sec:herramientas}

% aparte de las obvias (word, excel, ppt, latex, editor de video, etc.)

% -------- Neural Radiance Fields (NeRF) ------- %
\subsection{Neural Radiance Fields (NeRF)} \label{sec:herramientas:nerf}
% https://www.matthewtancik.com/nerf
% https://radiancefields.com/

\subsubsection{instant-ngp} \label{sec:herramientas:nerf:instant-ngp}
% https://nvlabs.github.io/instant-ngp/
% https://github.com/NVlabs/instant-ngp
% https://developer.nvidia.com/blog/getting-started-with-nvidia-instant-nerfs/

\subsubsection{nerfstudio} \label{sec:herramientas:nerf:nerfstudio}
% https://docs.nerf.studio/
% ---------------------------------------------- %


% -------- 3D Gaussian Splatting (3DGS) -------- %
\subsection{3D Gaussian Splatting (3DGS)} \label{sec:herramientas:3dgs}
% https://repo-sam.inria.fr/fungraph/3d-gaussian-splatting/

\subsubsection{nerfstudio} \label{sec:herramientas:3dgs:nerfstudio}
% ---------------------------------------------- %


% ------------- Unreal Engine (UE) ------------- %
\subsection{Unreal Engine (UE)} \label{sec:herramientas:ue}
% ---------------------------------------------- %


% ------------------ Volinga ------------------- %
\subsection{Volinga} \label{sec:herramientas:volinga}
% https://docs.nerf.studio/extensions/unreal_engine.html
% https://volinga.ai/
% https://github.com/Volinga/volinga-model
% ---------------------------------------------- %


% ----------------- VirtualBox ----------------- %
\subsection{VirtualBox} \label{sec:herramientas:virtualbox}
% ---------------------------------------------- %


% -------- Robot Operating System (ROS) -------- %
\subsection{Robot Operating System (ROS)} \label{sec:herramientas:ros}
% ---------------------------------------------- %


% --------------- Rosbridge --------------- %
\subsection{Rosbridge} 
\label{sec:herramientas:rosbridge}
% ---------------------------------------------- %


% --------------- ROSIntegration --------------- %
\subsection{ROSIntegration} \label{sec:herramientas:rosintegration}
% ---------------------------------------------- %


% ------------------ Blender ------------------- %
\subsection{Blender} \label{sec:herramientas:blender}
% ---------------------------------------------- %


% Termina la página actual y hace que se impriman todas las figuras y tablas que han aparecido hasta ahora en la entrada:
\clearpage

%%%%%%%%%%%%%%%%%%%%%%%%%%%%%%%%%%%%%%%%%%%%%%%%%%



%%%%%%%%%%%%%%% - IMPLEMENTACIÓN - %%%%%%%%%%%%%%%

\newpage
\section{IMPLEMENTACIÓN} \label{sec:implementacion}

\subsection{Líneas generales} 

o

\subsection{Breve resumen de la implementación} 

En un primer momento se prueba \gls{instant-ngp} a partir del paper de dicha investigación. 

Una vez generado el modelo 3D como nube de puntos la exportación se hace perdiendo demasiada información, se pierden las ventajas de esta nueva tecnología, así que se descarta este camino y se buscan otras opciones. 

Se prueba NerfStudio, que en conjunto con Volinga Suite permite generar una nube de puntos exportable a Unreal Engine 5 con resultado satisfactorio. Aunque tiene las siguientes desventajas: 

\begin{itemize} 
\item La nube de puntos no interactúa físicamente con los \gls{actores} dentro de \gls{unrealengine} (por ejemplo, el robot). La sensación visual es como si fuera una bruma, se puede atravesar. 

\item La integración con la realidad virtual no está optimizada, al visualizarlo con las gafas de realidad virtual se ve doble. 
\end{itemize} 

Se decide seguir adelante con el desarrollo usando la segunda opción: NerfStudio+Volinga durante un tiempo. 

Se consigue generar un \gls{modelo3d} básico del robot en \acrshort{ue} y controlar su movimiento mediante órdenes de tipo \gls{twist}. Estas órdenes se mandan mediante unas flechas flotantes dentro de la interfaz. 

Se consigue el control del robot real mediante el uso de la herramienta \gls{rosintegration} para \acrshort{ue} en conjunto con \gls{rosbridge} para la conexión con la máquina virtual donde se alojan los nodos de \acrshort{ros}. 

Más adelante Volinga lanza una versión alpha, a la cual esta investigación tiene acceso en exclusiva, en la que se integra la compatibilidad con la realidad virtual. Esta versión es compatible únicamente con \acrshort{3dgs}. 

Se decide probar \acrshort{3dgs} observándose las siguientes diferencias operativas de esta tecnología frente a \acrshort{nerf}: 

\begin{itemize}
\item En la optimización de los splats de \acrshort{3dgs} a veces no se obtienen resultados tan satisfactorios como en \acrshort{nerf}, pero el \gls{modelo3d} generado es de una calidad altamente satisfactoria. 

\item mientras que el tiempo de entrenamiento de la red neuronal de \acrshort{nerf} se cuenta en segundos (de 30 segundos a 2 minutos dependiendo del tamaño del modelo), para \acrshort{3dgs} el tiempo se cuenta en minutos (de 15 minutos a 1 hora). 

\item El rendimiento de los modelos generados por \acrshort{3dgs} es realmente superior a \gls{nerf}. La optimización está bien lograda consiguiendo una fluidez total en su utilización. 
\end{itemize} 

Se decide que esta última es la opción, 3DGS + Volinga, es la más completa y se decide terminar la implementación con esta configuración. 

El conjunto de herramientas elegido finalmente es: 

% Añadir diagrama de bloques de las herramientas para cada parte

\begin{itemize} 
\item 3DGS generado con NerfStudio aplicando la Suite de Volinga en su versión alpha compatible con VR para la generación del objeto 3D. 

\item Unreal Engine con los plugin de Volinga y ROSIntegration para su compatibilidad con el objeto generado y con su comunicación con ROS. 

\item La máquina virtual de ubuntu con Rosbridge para la comunicación con su host de Windows, que contiene los nodos necesarios para crear la conexión entre los topics de UE y el robot real, conectada en modo ROS slave al robot real mediante red Wi-Fi local. 
\end{itemize} 

El diagrama de bloques del flujo de trabajo final es el siguiente
% Añadir diagrama de bloques del flujo de trabajo

En las siguientes subsecciones se explica el proceso de implementación con mayor nivel de detalle. 

% listar subsecciones antes de empezar con ello? 

% ir en orden de cómo fui haciendo las cosas y contar lo más importante de mi desarrollo 

% hacer referencia a los manuales (publicarlos en github?) 

% aquí empieza instant-ngp, ver como lo meto luego con lo que ya tengo 

\subsection{Instant-ngp de NVIDIA} 

En primer lugar, se prueba la tecnología que hizo comenzar la presente investigación: \gls{instant-ngp}. 

Se usa como base el paper original y las instrucciones en el \href{https://github.com/NVlabs/instant-ngp}{repositorio oficial}.  

La puesta en marcha de \gls{instant-ngp} desarrollado por \gls{nvidia} requiere de un entorno con paquete de desarrollo de motor gráfico CUDA 11.8 y toda la configuración que ello conlleva. Se decide crear el entorno en Ubuntu 18.04 por compatibilidad con ROS y las líneas de investigación actuales del laboratorio. 

Se agregan todos los programas y librerías de dependencias para la instalación de CUDA así como todos los requisitos posteriores a la instalación como por ejemplo la prueba de los cuda-samples para verificar su correcto funcionamiento. 

Además, para la interpretación de la fotogrametría de datasets propios que incluye el flujo de trabajo de instant-ngp en su aplicación de Python es necesario un programa externo. Los autores del paper recomiendan Colmap y ffmpeg. Se instalan y agregan sus dependencias. 

Por último, se descarga el código fuente de instant-ngp del GitHub de NVIDIA NVlabs que es open source. Una vez compilado e instaladas las dependencias ya está listo para usar. 

Para más detalle sobre la instalación ver anexos de configuración del entorno. 

Una vez configurado se puede lanzar el script de Python proporcionado en el repositorio de instant-ngp que transforma el vídeo directamente en su modelo 3D. Este hace uso ffmpeg para extraer los fotogramas del vídeo formando un dataset de imágenes y de Colmap para obtener la fotogrametría de dicho dataset.  

% Añadir diagrama del proceso

La fotogrametría junto con el dataset de imágenes es el input de la red neuronal que en segundos (entre 30 segundos y un minuto) es capaz de generar un modelo 3D de alta fidelidad del objeto o entorno de estudio. 

% insertar tablas de velocidades de generación para distintos dataset?  

Por desgracia la generación de la fotogrametría es muy costosa para la GPU y requiere de largos tiempos dependiendo de la cantidad de imágenes a tratar. El entrenamiento de la red neuronal también usa la GPU para sus cálculos.

Se procede a lanzar de nuevo la herramienta de instant-npg de NVIDIA. Esta tiene un visor gráfico 3D para observar los resultados del entrenamiento de la red en tiempo real y visualizar cómo la nube de puntos va convergiendo en la imagen de nuestro dataset. Una vez obtenido un resultado aceptable se puede detener el entrenamiento y perfilar el modelo generado, recortando la parte deseada, por ejemplo. Además, se puede usar el visualizador para comprobar la calidad del dataset y generar uno nuevo si fuera necesario.

Moverse por dentro del visor 3D para observar el objeto desde diferentes ángulos requiere mucha potencia de computación y se experimenta una falta de fluidez debido a la capacidad limitada de los sistemas disponibles en el laboratorio. Aun así, se comprueba que el modelo generado es de gran calidad y extremadamente realista. Tampoco tiene problemas para representar superficies reflectantes o transparentes.

%como generar el modelo 3D de algo 

Una vez seleccionada un área de recorte (que puede incluir al modelo entero) se puede exportar la \gls{nubedepuntos} en forma de objeto sólido o \gls{vertexmesh} interpretable por la mayoría de los \gls{motoresgraficos3d} del mercado como \gls{unity} o \gls{unrealengine}. 

Los dos principales problemas de este tipo de exportación de objeto sólido son: 

\begin{itemize} 
\item Para generar los planos que conforman las superficies de los objetos de tipo \gls{vertexmesh} a partir de la \gls{nubedepuntos}  se requiere una gran capacidad de computación. Esto limita la resolución máxima a la que se puede exportar el objeto, es decir, el número máximo de vértices que contiene, que depende de la capacidad de procesamiento la unidad gráfica de nuestro ordenador (VERIFICAR ESTA INFO). 

\item La pérdida de la propiedad del color. El estándar del archivo tipo .obj que contiene el \gls{vertexmesh} incluye información de color, sólo de la posición de los vértices. Aunque el archivo generado por instant-ngp sí guarda información sobre el color, ésta sólo es interpretada por algunos motores gráficos como MeshLab, pero no por la mayoría ni por los que se usan en el desarrollo de aplicaciones de robótica. 
\end{itemize} 

Las limitaciones de los ordenadores convencionales impiden una exportación de mayor resolución que implicaría miles de millones de vértices. Para objetos individuales se puede alcanzar una resolución aceptable, pero para un entorno como una habitación o una zona exterior la pérdida de información se multiplica y se convierte en inaceptable, generándose volúmenes que poco se parecen a la realidad. 

% insertar ejemplos de visualización de mesh en Unity en blanco y negro y en baja resolución 

% añadir también algún ejemplo de visualización mejor por ejemplo del usb de la zapatilla 

Se intenta también la exportación de la nube de puntos de forma directa, pero los motores gráficos compatibles con aplicaciones de robótica no interpretan ese tipo de archivos y se acaba desechando la idea.

Debido a la importancia que tiene la GPU en el desempeño de esta investigación, se decide reemplazar la tarjeta gráfica del ordenador por una NVIDIA GeForce RTX 2060 12GB más potente. También se decide añadir un disco duro SSD nuevo para tener más espacio para la partición de Ubuntu 18.04. Esto resulta en la reinstalación de todo el entorno desde cero.

Las mejoras de hardware producen cambios sustanciales en la velocidad de cálculo de la fotogrametría de los dataset así como del entrenamiento de la red neuronal. También se experimenta una mejora de la fluidez en el visor 3D de la herramienta. Sin embargo, no se aprecian mejoras suficientes en la exportación del modelo.

En vista de la pérdida de resolución y color que se produce durante la exportación y que el objeto vertex mesh obtenido no supone en la práctica una mejora real con respecto a otros modelos que usaban fotogrametría anteriormente se descarta el uso de instant-ngp como herramienta para la generación de entornos 3D para su uso en el control de robots mediante realidad virtual inmersiva.

 % mencionar comparativa con fotogrametría actual ?? Se ve similar en cuanto a pataterismo Ej: https://www.youtube.com/watch?v=-1Dz9Izq8Vs min 0:30  

\subsection{NerfStudio + Volinga}

El propósito de NerfStudio es acercar la tecnología NeRF al público de la manera más sencilla posible. Se trata de una herramienta opensource que crea un marco de trabajo modular de fácil integración de forma que simplifica el proceso de crear, entrenar y testear NeRFs (\cite{nerfstudio}). \href{https://docs.nerf.studio/index.html}{Su completa y estructurada web} dispone de \href{https://docs.nerf.studio/quickstart/installation.html}{guía de instalación} así como \href{https://docs.nerf.studio/quickstart/custom_dataset.html}{instrucciones detalladas de cómo usar dataset propios}.   

Para integrar esta opción se elige en un primer momento aprovechar el entorno ya configurado para instant-ngp en Ubuntu 18.04. Siguiendo las guías mencionadas en el párrafo anterior se consigue generar un resultado similar al de instant-ngp a partir del mismo input.  

El proceso desde el punto de vista del usuario es similar, primero se procesa el archivo de origen para generar un dataset de imágenes con ffmpeg. A continuación, se obtiene la fotogrametría de las imágenes con Colmap. Para terminar, esa información se usa para entrenar la red neuronal y generar el modelo 3D. 

%Añadir diagrama del proceso 

 Por ahora se detectan dos grandes diferencias: 

\begin{itemize} 

\item La sencillez de las instrucciones, requisitos y pasos necesarios para la instalación de NerfStudio frente a instant-ngp. 

\item El visualizador de NerfStudio en el navegador web es mucho más cómodo y amable con el usuario que el visualizador de instant-ngp. Además, se detecta mayor fluidez a la hora de manejarlo. 

\end{itemize} 

Para la exportación a Unreal Engine es necesario entrenar la red neuronal con unos parámetros determinados, esto nos permitirá generar los archivos de tipo checkpoint de NerfStudio (.ckpt) en un formato compatible con la herramienta Volinga Suite. 

Para ello es necesario instalar el modelo de Volinga (volinga-model) que usará NerfStudio para esta generación especial. Este modelo es también opensource y está disponible para descargar en el \href{https://github.com/Volinga/volinga-model}{repositorio de GitHub de volinga-model} puesto a disposición por la compañía. Volinga-model es compatible con todo el entorno usado hasta el momento en Ubuntu 18.04. 

En el momento de realizar el siguiente paso por primera vez la herramienta de Volinga Suite que interpreta los archivos checkpoint de NerfStudio generados con los parámetros de volinga-model está en fase de desarrollo. Mediante el formulario en su página web se hace una petición de acceso anticipado para su uso en investigación uniersitaria.  

Una vez concedido el acceso se puede usar la herramienta de Volinga Suite con su aplicación de escritorio o directamente a través de su aplicación web. Con ella se genera el objeto de tipo .nvol compatible con Unreal Engine. Para la interpretación del archivo .nvol UE necesita además el plugin de Volinga para UE, al cual se concede acceso junto a Volinga Suite. 

Para la instalación de Unreal Engine en Ubuntu es necesaria la descarga y compilación del código fuente de la aplicación. Se pide acceso al panel de desarrolladores de Unreal Engine en la zona de registro de organizaciones de GitHub. Una vez concedido se descarga el código fuente y se genera el entorno de trabajo para UE. 

La última versión compatible de UE con Ubuntu 18.04 (el entorno de NeRF) es la 4.27, pero el plugin de Volinga Suite sólo es compatible con Unreal Engine a partir de la versión 5.0 (UE5). 

El motor de UE5 requiere de una versión mínima de Ubuntu 20.04. por lo que se debe generar de nuevo todo el entorno con la instalación de Cuda 11.8 y las herramientas necesarias para que funcione NerfStudio como Colmap y ffmpeg. 

Lamentablemente Cuda 11.8 ya no es compatible con con Ubuntu 20.04. La versión mínima compatible de los drivers de NVIDIA con Ubuntu 20.04. es la 525, y Cuda 11.8 requiere de los drivers 520. Por tanto, se decide que, por el momento, el entrenamiento de la red con NerfStudio se haga en Ubuntu 18.04., mientras que la instalación de UE5 y la posterior integración con ROS se hará en la nueva partición que se ha creado para Ubuntu 20.04. 

A la hora de instalar el plugin de Volinga en UE a fin de que éste interprete el objeto de tipo .nvol generado por la aplicación web de Volinga se descubre que éste no soporta un entorno Linux.  

Por desgracia al estar la herramienta en desarrollo los requerimientos mínimos también lo están. Por suerte esto desencadena una conversación fluida y muy fructífera con los fundadores de la compañía. 

Finalmente se instala UE5 en Windows y se mantiene el fljo de trabajo de NerfStudio en Ubuntu 18.04. 

Con el plugin de Volinga instalado en UE5 se puede agregar directamente el objeto .nvol y visualizar un modelo 3D muy similar al que se genera en la red neuronal original de NerfStudio previo al paso por la aplicación web de Volinga. Dentro de UE se observa una cierta pérdida de resolución y pequeños fallos en la interpretación de la escena respecto del modelo generado por instant-ngp. Esto es debido a pequeñas optimizaciones llevadas a cabo por NerfStudio durante el entrenamiento de la red.  

Además, surge un nuevo problema derivado de la intensidad en el consumo de recursos de computación de la CPU por parte de UE. UE usa la GPU para el renderizado de gráficos, pero para el cálculo de las sombras, las texturas y la interpretación de la nube de puntos UE usa la CPU. Esto hace que la simulación con el ordenador no sea fluida. 

En cualquier caso, el resultado se considera válido ya que supone una mejora enorme en cuanto a realismo respecto a tecnologías previas. 

%desarrollo del Digital Twin 

Los gemelos digitales disponibles en el laboratorio Robcib están desarrollados en Unity. Así pues, llegado este punto, hace falta desarrollar un nuevo gemelo digital del robot cuadrúpedo \href{https://www.unitree.com/a1/}{Unitree A1} con el que se probará esta tecnología en Unreal Engine. 

Para ello se usa la herramienta Blender y se utiliza el \herf{https://github.com/unitreerobotics/unitree_cad/tree/main/a1}{modelo CAD} proporcionado por la empresa fabricante del robot \href{https://www.unitree.com/}{Unitree Robotics}.  

El modelo CAD contiene las piezas del robot por separado y con Blender se realiza la unión de éstas mediante un esqueleto o rig. El uso de un rig permite generar distintas poses para el robot y desarrollar, si se desea, animaciones para su movimiento dentro del motor gráfico 3D. 

Una vez importado el modelo CAD con la pose deseada en Unreal Engine se crea un actor dentro del proyecto de UE al cual se asigna el mesh del modelo CAD.  

El control de dicho actor se realiza programando con blueprints. Se crea un widget con botones de flechas cuyos eventos de en clicado desencadenan las funciones de los movimientos del robot. 

Se demuestra que el robot se puede controlar dentro de la simulación en un proyecto de UE de simulación con blueprints. 

%ROSIntegration 

Una vez conseguido lo anterior se intenta la conexión con ROS. Para Unreal Engine se hace uso de una herramienta desarrollada por el Institute for Artificial Intelligence de la University of Bremen llamada ROSIntegration. Esta herramienta está diseñada en forma de plugin para Unreal Engine 4, pero dicho plugin se puede recompilar en un proyecto de UE5 en C++ para que funcione correctamente en éste.



Para la comunicación con los robots se utiliza ROS, que es el estándar para aplicaciones de robótica. ROS está optimizado para funcionar en Ubuntu y su instalación en Windows es problemática, es por ello que para permitir la comunicación entre UE5 y ROS instalaremos Ubuntu 20.04. en una máquina vitual del VirtualBox 7 y realizaremos la conexión con el host de Windows mediante SSH. La conexión SSH se hace por defecto a través del puerto 22 de la máquina virtual, pero para no usar un puerto reservado la realizaremos a través de un puerto no reservado (entre el 1024 y el 65536) por ejemplo el puerto 2222.

Para que funcione la conexión con ROS debemos instalar ROSBridge en nuestra máquina virtual y habilitar otro puerto para la conexión con el host de Windows. Se usa por convenio el puerto 9090. Debe lanzarse ROSBridge con una configuración específica para funcionar con ROSIntegration. Una vez establecida la conexión podemos publicar y suscribirnos a los topics o services de ROS que queramos a ambos lados de la conexión.

Comprobamos que la comunicación entre ROS y UE5 a través de ROSBridge y ROSIntegration funciona correctamente.

Por desgracia ROSIntegration sólo tiene implementada la suscripción a los mensajes estándar de ROS,

TBD:

A continuación, hablar de ROSIntegration e incompatibilidad, cómo hacer los topic y si se pueden modificar

Hablar de los problemas generando la fotogrametría con colmap (es muuy puñetero)

Ver si se puede probar a usar algo de luma labs en el iphone de alguien para que la fotogrametría venga hecha de “fábrica”

%Añadir esto como transición de nerf a 3dgs (?): 

Esto ya se ha comentado antes pero hilarlo de alguna forma {Sin embargo, surge un nuevo problema derivado de la intensidad en el consumo de recursos de computación de la CPU por parte de UE. UE usa la GPU para el renderizado de gráficos, pero para el cálculo de las sombras, las texturas y la interpretación de la nube de puntos UE usa la CPU. Esto hace que la simulación con el ordenador no sea fluida,} impidiendo una evaluación correcta del uso conjunto del Digital Twin del robot con el modelo 3D generado por Volinga. 

Para solucionarlo se aplicarán diversas optimizaciones en los gráficos de UE sugeridas en el foro de Discord de Volinga como eliminar el suelo, el Sol y todos los elementos que generan interacciones físicas y de sombras salvo lo estrictamente necesario: el gemelo digital del robot y la nube de puntos. 

\subsection{3DGS + Volinga}

% Termina la página actual y hace que se impriman todas las figuras y tablas que han aparecido hasta ahora en la entrada:
\clearpage

%%%%%%%%%%%%%%%%%%%%%%%%%%%%%%%%%%%%%%%%%%%%%%%%%%



%%%%%%%%%%% - RESULTADOS Y DISCUSIÓN - %%%%%%%%%%%

\newpage
\section{RESULTADOS Y DISCUSIÓN} \label{sec:resultados_y_discusion}

% Termina la página actual y hace que se impriman todas las figuras y tablas que han aparecido hasta ahora en la entrada:
\clearpage

%%%%%%%%%%%%%%%%%%%%%%%%%%%%%%%%%%%%%%%%%%%%%%%%%%



%%%%%%%%%%%%%%%% - CONCLUSIONES - %%%%%%%%%%%%%%%%

\newpage
\section{CONCLUSIONES} \label{sec:conclusiones}

% Termina la página actual y hace que se impriman todas las figuras y tablas que han aparecido hasta ahora en la entrada:
\clearpage

%%%%%%%%%%%%%%%%%%%%%%%%%%%%%%%%%%%%%%%%%%%%%%%%%%



%%%%%%%%%%%%%%% - LÍNEAS FUTURAS - %%%%%%%%%%%%%%%

\newpage
\section{LÍNEAS FUTURAS} \label{sec:lineas_futuras}

% Termina la página actual y hace que se impriman todas las figuras y tablas que han aparecido hasta ahora en la entrada:
\clearpage

%%%%%%%%%%%%%%%%%%%%%%%%%%%%%%%%%%%%%%%%%%%%%%%%%%



%%%%%%%%%%%%%%%% - BIBLIOGRAFÍA - %%%%%%%%%%%%%%%%

\newpage
% Se genera la bibliografía mediante el comando \printbibliography (en ella aparecen únicamente las referencias citadas a lo largo del documento):
\appto{\bibsetup}{\sloppy}
\printbibliography[heading=bibintoc, title=BIBLIOGRAFÍA] % el argumento "title" puede modificarse indicando el título que convenga (bibliografía, referencias, etc.).

% Termina la página actual y hace que se impriman todas las figuras y tablas que han aparecido hasta ahora en la entrada:
\clearpage

%%%%%%%%%%%%%%%%%%%%%%%%%%%%%%%%%%%%%%%%%%%%%%%%%%








%%%%%%%%%%%%%%%%%%% - ANEXOS - %%%%%%%%%%%%%%%%%%%
% FALTA NUMERARLOS Y PONER LOS TÍTULOS EN MINÚSCULAS
% (los índices de tablas, figuras y códigos son anexos y diría que las licencias también)

\newpage

\section*{ANEXOS} \label{sec:anexos} % Se añade un asterisco a \section para que el título no esté numerado.
\addcontentsline{toc}{section}{ANEXOS} % Al utilizar \section* se ha de añadir manualmente el apartado al índice (Table Of Contents, TOC).
\markright{ANEXOS} % Al utilizar \section* se ha de añadir manualmente el título del apartado al encabezado.

\renewcommand{\thesubsection}{\Alph{subsection}} % Se numeran los anexos con letras del alphabeto en lugar de números.
% Se indica que las tablas, figuras y códigos se numeran con el código del anexo (A, B, C, ...) seguido del número de tabla, figura o código dentro del anexo (tabla A.2, figura C.1, etc.)
\renewcommand{\thetable}{\Alph{subsection}.\arabic{table}}
\renewcommand{\thefigure}{\Alph{subsection}.\arabic{figure}}
\renewcommand{\thecode}{\Alph{subsection}.\arabic{code}}

% ---------------- Primer anexo ---------------- %
%%%%%%%%%%% - EVALUACIÓN DE IMPACTOS - %%%%%%%%%%%

\subsection{Anexo I: Evaluación de impactos} \label{sec:anexo1}
\subsubsection{Impacto ambiental} \label{sec:anexo1:ambiental}
\subsubsection{Impacto social} \label{sec:anexo1:social}
\subsubsection{Impacto ético} \label{sec:anexo1:etico}
\subsubsection{Impacto legal} \label{sec:anexo1:legal}
\subsubsection{Contribución a los Objetivos de Desarrollo Sostenible (ODS)} \label{sec:ods}

% Termina la página actual y hace que se impriman todas las figuras y tablas que han aparecido hasta ahora en la entrada:
\clearpage

%%%%%%%%%%%%%%%%%%%%%%%%%%%%%%%%%%%%%%%%%%%%%%%%%%
% ---------------------------------------------- %


% ---------------- Segundo anexo --------------- %
%%% - PLANIFICACIÓN TEMPORAL Y PRESUPUESTOS - %%%%

\newpage
\subsection{Anexo II: Planificación temporal y presupuestos} \label{sec:anexo2}
\subsubsection{Planificación temporal} \label{sec:anexo2:temporal}
\subsubsection{Presupuestos} \label{sec:anexo2:presupuestos}

% Termina la página actual y hace que se impriman todas las figuras y tablas que han aparecido hasta ahora en la entrada:
\clearpage

%%%%%%%%%%%%%%%%%%%%%%%%%%%%%%%%%%%%%%%%%%%%%%%%%%
% ---------------------------------------------- %


% ---------------- Tercer anexo ---------------- %
%%%%%%%%%%%% - GLOSARIO Y ACRÓNIMOS - %%%%%%%%%%%%

\newpage
\subsection{Anexo III: Glosario y acrónimos} \label{sec:anexo3}

\glsaddall

% Hacer más grande el tamaño del título de la subsubseccion glosario y acronimos
%\renewcommand{\subsubsectionsize}{\large}

\printglossary[title=Glosario, toctitle=Glosario]

\printglossary[type=\acronymtype, title=Acrónimos, toctitle=Acrónimos]

%\renewcommand{\subsubsectionsize}{\normal}

% Termina la página actual y hace que se impriman todas las figuras y tablas que han aparecido hasta ahora en la entrada:
\clearpage

%%%%%%%%%%%%%%%%%%%%%%%%%%%%%%%%%%%%%%%%%%%%%%%%%%
% ---------------------------------------------- %


% ---------------- Cuarto anexo ---------------- %
%%%%%%%%%%%%%%%%% - LICENCIAS - %%%%%%%%%%%%%%%%%%

\newpage
\subsection{Anexo IV: Licencias} \label{sec:anexo4}

Licencia de instant-ngp: https://github.com/NVlabs/instant-ngp/blob/master/LICENSE.txt y al final de https://github.com/NVlabs/instant-ngp

Licencia de NeRF: https://github.com/bmild/nerf/blob/master/LICENSE

Licencia de NerfStudio: https://github.com/nerfstudio-project/nerfstudio/blob/main/LICENSE

Licencia ROSIntegration: https://github.com/code-iai/ROSIntegration/blob/master/LICENSE

\begin{wrapfigure}{L}{0.21\textwidth}
    \vspace{-\baselineskip}
    \href{http://creativecommons.org/licenses/by/4.0/}{\includegraphics[width=0.22\textwidth]{cc-by.png}}
\end{wrapfigure} 

\vspace*{\fill}
``Plantilla en LaTeX acorde con la Normativa para la elaboración de informes de TFT de la ETSII (UPM)" \ by Javier Soto Pérez-Olivares is licensed under a \href{http://creativecommons.org/licenses/by/4.0/}{Creative Commons Attribution 4.0 International License}.

% Termina la página actual y hace que se impriman todas las figuras y tablas que han aparecido hasta ahora en la entrada:
\clearpage

% ---------------------------------------------- % 
%%%%%%%%%%%%%%%%%%%%%%%%%%%%%%%%%%%%%%%%%%%%%%%%%%
% ---------------------------------------------- %


%%%%%%%%%%%%%% - FIN DEL DOCUMENTO - %%%%%%%%%%%%%

\end{document}

%%%%%%%%%%%%%%%%%%%%%%%%%%%%%%%%%%%%%%%%%%%%%%%%%%