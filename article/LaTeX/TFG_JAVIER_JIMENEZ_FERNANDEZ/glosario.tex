
%%%%%%%%%%%%%%%%% - GLOSARIO - %%%%%%%%%%%%%%%%%%%

% -------------------- 0 - 9 ------------------- %

\newglossaryentry{3dgaussiansplatting}
{
    name={\acrshort{3d} Gaussian Splatting},
    description={TBD}
}

% ---------------------- A --------------------- %

\newglossaryentry{actor}
{
    name={actor},
    description={TBD}
}

\newglossaryentry{actores}
{
    name={actores},
    description={. Ver \gls{actor}}
}

% ---------------------- B --------------------- %

\newglossaryentry{blender}
{
    name={Blender},
    description={TBD}
}

% ---------------------- C --------------------- %

\newglossaryentry{cudarchitecture}
{
    name={Compute Unified Device Architecture},
    description={o Arquitectura Unificada de Dispositivos de Cómputo}
}

% ---------------------- D --------------------- %

\newglossaryentry{digitaltwin}
{
    name={Digital Twin},
    description={o gemelo digital es una representación digital de un objeto, generalmente un \gls{modelo3d}. En robótica y automatización se usa para realizar versiones virtuales de robots o máquinas para simular su funcionamiento antes de su puesta en marcha o durante su operación.}
}

% ---------------------- E --------------------- %

% ---------------------- F --------------------- %

\newglossaryentry{fotogrametria}
{
    name={fotogrametría},
    description={es el método destinado a analizar y determinar con exactitud la geometría, tamaño y ubicación espacial de un objeto, empleando principalmente mediciones realizadas a partir de una o más imágenes fotográficas de dicho objeto.}
}

% ---------------------- G --------------------- %

% ---------------------- H --------------------- %

% ---------------------- I --------------------- %

\newglossaryentry{instant-ngp}
{
    name={instant-ngp},
    description={TBD. (\cite{mueller2022instant})}
}

\newglossaryentry{inteligenciaartificial}
{
    name={Inteligencia Artificial},
    description={TBD}
}

% ---------------------- J --------------------- %

% ---------------------- K --------------------- %

% ---------------------- L --------------------- %

% ---------------------- M --------------------- %

\newglossaryentry{mesh}
{
    name={mesh},
    description={TBD}
}

\newglossaryentry{modelo3d}
{
    name={modelo \acrshort{3d}},
    description={TBD}
}

\newglossaryentry{motoresgraficos3d}
{
    name={motores gráficos \acrshort{3d}},
    description={. Ver \gls{motorgrafico3d}}
}

\newglossaryentry{motorgrafico3d}
{
    name={motor gráfico \acrshort{3d}},
    description={TBD}
}

% ---------------------- N --------------------- %

\newglossaryentry{nerfstudio}
{
    name={NerfStudio},
    description={TBD. (\cite{nerfstudio})}
}

\newglossaryentry{neuralradiancefields}
{
    name={Neural \glspl{radiancefield}},
    description={TBD. (\cite{mildenhall2020nerf})}
}

\newglossaryentry{nodo}
{
    name={nodo},
    description={TBD}
}

\newglossaryentry{nubedepuntos}
{
    name={nube de puntos},
    description={TBD}
}

\newglossaryentry{nvidia}
{
    name={NVIDIA},
    description={NVIDIA Corporation}
}

\newglossaryentry{nvlabs}
{
    name={NVlabs},
    description={TBD}
}

% ---------------------- O --------------------- %

% ---------------------- P --------------------- %

% ---------------------- Q --------------------- %

% ---------------------- R --------------------- %

\newglossaryentry{radiancefield}
{
    name={Radiance Field},
    description={TBD}
}

\newglossaryentry{redneuronal}
{
    name={red neuronal},
    description={TBD}
}

\newglossaryentry{renderizar}
{
    name={renderizar},
    description={TBD}
}

\newglossaryentry{renderizado}
{
    name={renderizado},
    description={TBD}
}

\newglossaryentry{robotoperatingsystem}
{
    name={Robot Operating System},
    description={TBD}
}

\newglossaryentry{rosbridge}
{
    name={Rosbridge},
    description={TBD}
}

\newglossaryentry{rosintegration}
{
    name={ROSIntegration},
    description={TBD}
}

% ---------------------- S --------------------- %

% ---------------------- T --------------------- %

\newglossaryentry{topic}
{
    name={topic},
    description={TBD}
}

\newglossaryentry{twist}
{
    name={twist},
    description={TBD}
}

% ---------------------- U --------------------- %

\newglossaryentry{unity}
{
    name={Unity},
    description={TBD}
}

\newglossaryentry{unrealengine}
{
    name={Unreal Engine},
    description={TBD}
}

% ---------------------- V --------------------- %

\newglossaryentry{vertexmesh}
{
    name={vertex \gls{mesh}},
    description={TBD}
}

\newglossaryentry{virtualbox}
{
    name={VirtualBox},
    description={TBD}
}

\newglossaryentry{volinga}
{
    name={Volinga},
    description={TBD. Véase también \gls{volingasuite}}
}

\newglossaryentry{volingasuite}
{
    name={\gls{volinga} Suite},
    description={TBD. software de la empresa Volinga, muchas veces referido en este documento directamente como \gls{volinga}}
}

% ---------------------- W --------------------- %

% ---------------------- X --------------------- %

% ---------------------- Y --------------------- %

% ---------------------- Z --------------------- %

%%%%%%%%%%%%%%%%%%%%%%%%%%%%%%%%%%%%%%%%%%%%%%%%%%